\documentclass{article}
\usepackage[utf8]{inputenc}

\usepackage{natbib}
\usepackage{graphicx}

\usepackage[slovene]{babel}
\usepackage{amsmath}
\usepackage{amsfonts}
\usepackage{amssymb}

\usepackage[T1]{fontenc}
\usepackage{listingsutf8}
\lstset{literate={č}{{\v c}}1 {š}{{\v s}}1 {ž}{{\v z}}1}

\usepackage{algorithm}
\usepackage{algorithmicx}
\usepackage{algpseudocode}

\usepackage{tikz}
\usetikzlibrary{graphs,quotes,arrows.meta}
\usetikzlibrary{positioning}


\title{3. domača naloga}
\author{Lovro Habjan}
\date{\today}


\begin{document}

\maketitle

\section{Problem 1}

\subsection{Podproblem A}

Napiši algoritem, ki izračuna unijo dveh konveksnih ovojnic. - naredi samo da vzameš točke obeh in poženeš navaden algoritem

\subsection{Podproblem B}

Napiši algoritem, ki preveri, če se konveksni ovojnici sekata. - Jordan curve teorem - neskončen poltrak - kolikokrat seka poltrak

\subsection{Podproblem C}

Implementacija algoritmov.

\section{Problem 2}

\section{Problem 3}

\end{document}